\documentclass[journal,twoside]{JoPhA}
\usepackage[utf8]{inputenc}
\usepackage{flushend}
\usepackage{graphicx}


\begin{document}

\title{Predicción de mortalidad en accidentes de tráfico}
 
\author{Samuel Rey y David Moreno
\IEEEcompsocitemizethanks{\IEEEcompsocthanksitem Samuel Rey and Jos\'e M. Ca\~nas are with Universidad Rey Juan Carlos\protect\\

E-mail: samuel.rey.escudero@gmail.com, dmorenolumb@gmail.com.es
%\IEEEcompsocthanksitem Vicente Matell\'an is with University of Rey Juan Carlos.
} % <-this % s
}


\maketitle


\begin{abstract}
% Yo
	
Resumen
\end{IEEEkeywords}


\section{Introduction}

% David

% TODO: Comentar referencia
% TODO: Hablar del conjunto de datos
\IEEEPARstart{E}l problema expuesto en este documento corresponde dentro del ámbito de aprendizaje máquina (o \textit{Machine Learning}) como un problema de clasificación, concretamente para la predicción de la mortalidad de accidentes de tráfico. La predicción de los accidentes de tráfico es uno de los mayores retos de la actual sociedad debido a su alto coste humano y económico asociado. Se analizarán y utilizarán distintas técnicas de análisis de datos y aprendizaje máquina para entender, analizar e intentar obtener una predicción de accidentes.

El analisis y estudio de estos datos para la predicción es una tarea compleja por diferentes factores: las bases de datos son muy grandes y heterogéneas, los datos son registrados de forma manual por autoridades [INCLUIR REFERENCIA AL MODELO DEL FORMULARIO???] y son altamente propensos a errores o valores inadecuados (outliers, NaN, etc.). [REFERENCIA AL 1] \\

El conjunto de datos del problema corresponde a una base de datos de más de 9000 entradas y más de 80 variables, cada variable corresponde a un apartado del forumlario. El conjunto de datos está dividido en un conjunto de entrenamiento y un conjunto de test. El conjunto de entrenamiento tiene un tamaño de 9000 entradas frente a las 200 entradas del conjunto de test.\\

Por lo tanto, el objetivo de este problema es predecir la gravedad de un accidente a partir de este conjunto de datos. Como se ha comentado anteriormente el problema corresponde a un problema de clasificación binaria con la siguiente distribución de clases: la gravedad del accidente se clasifica como 0, cuando no ha habido muertes ni heridos de ningún tipo; o como 1, cuando ha habido algún herido o muerte. Se analizarán distintos tipos de técnicas y modelos de análisis de datos, entre ellos, visualiación gráfica de los datos, información de las características, extracción de características, redes neuronales, etc. con el fin de obtener un modelo que proporcione la mejor clasificación.



\IEEEPARstart{I}ntroducción

\section{Metodología}
% Yo
\IEEEPARstart{I}ntroducción


\section{Resultados}
\IEEEPARstart{I}ntroducción
	\subsection{Datatón URJC 2017}
	% David

	\subsection{Definición alternativa de clases}
	% Yo


\section{Conclusiones}
% David

\section*{Acknowledgment??}
This  research  has  been  partially  sponsored  by  the Community of Madrid through the RoboCity2030-III project (S2013/MIT-2748), by the Spanish Ministerio de Economía y Competitividad through the SIRMAVED project (DPI2013-40534-R) and by the URJC-BancoSantander.
%CVIP.

\begin{thebibliography}{1}

% \bibitem{IEEEhowto:kopka}
% H.~Kopka and P.~W. Daly, \emph{A Guide to \LaTeX}, 3rd~ed.\hskip 1em plus
%   0.5em minus 0.4em\relax Harlow, England: Addison-Wesley, 1999.

\bibitem{intro}
Figuera, C., Lillo, J. M., Mora-Jimenez, I., Rojo-\'Alvarez, J. L., & Caama\~no, A. J. , \emph{Multivariate spatial clustering of traffic accidents for local profiling of risk factors}.  In Intelligent Transportation Systems (ITSC), 2011 14th International IEEE Conference on (pp. 740-745). IEEE.



\end{thebibliography}
\end{document}


